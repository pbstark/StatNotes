\date{DRAFT--Revised 1 May 2008}
% last modified 11/2/10 by PB Stark, statistics.berkeley.edu/~stark
\documentclass[12pt]{article}
\topmargin -.5in
\oddsidemargin -.4in
\evensidemargin -.4in
\textwidth 7.0in
\textheight 9in

\title{Statistics 240 Lecture Notes}
\author{P.B. Stark
{\tt statistics.berkeley.edu/$\sim$stark/index.html}}
\newfont{\msbm}{msbm10 at 12pt}
\renewcommand{\baselinestretch}{1.5}
\newcommand{\EE}{\mbox{\msbm E}}
\newcommand{\card}{\mbox{\#}}
\newcommand{\diam}{{\rm diam}}
\newcommand{\sgn}{{\rm sgn}}
\newcommand{\med}{{\rm med}}
\newcommand{\MAD}{{\rm MAD}}
\newcommand{\Prob}{{\bf P}}
\newcommand{\calGbar}{{\bar{{\cal G}}}}
\newcommand{\calA}{{\cal A}}
\newcommand{\calB}{{\cal B}}
\newcommand{\calC}{{\cal C}}
\newcommand{\cC}{{\cal C}}
\newcommand{\calD}{{\cal D}}
\newcommand{\cD}{{\cal D}}
\newcommand{\calF}{{\cal F}}
\newcommand{\calG}{{\cal G}}
\newcommand{\calH}{{\cal H}}
\newcommand{\cH}{{\cal H}}
\newcommand{\calI}{{\cal I}}
\newcommand{\calJ}{{\cal J}}
\newcommand{\calK}{{\cal K}}
\newcommand{\calL}{{\cal L}}
\newcommand{\calM}{{\cal M}}
\newcommand{\calN}{{\cal N}}
\newcommand{\calP}{{\cal P}}
\newcommand{\calQ}{{\cal Q}}
\newcommand{\calR}{{\cal R}}
\newcommand{\calS}{{\cal S}}
\newcommand{\calT}{{\cal T}}
\newcommand{\calU}{{\cal U}}
\newcommand{\calV}{{\cal V}}
\newcommand{\calW}{{\cal W}}
\newcommand{\calX}{{\cal X}}
\newcommand{\cX}{{\cal X}}
\newcommand{\calY}{{\cal Y}}
\newcommand{\calZ}{{\cal Z}}
\newcommand{\bfC}{{\bf C}}
\newcommand{\bfQ}{{\bf Q}}
\newcommand{\bfone}{{\bf 1}}
\def\Real{\hbox{I\kern-.17em\hbox{R}}}
\def\Prob{\hbox{I\kern-.17em\hbox{P}}}
\def\Expect{\hbox{I\kern-.17em\hbox{E}}}
\newcommand{\bfR}{\Real}
\newcommand{\bfRn}{\bfR^n}
\newcommand{\bfRN}{\bfR^N}
\newcommand{\bfRm}{\bfR^m}
\def\bfN{\hbox{I\kern-.17em\hbox{N}}}
\newcommand{\bfZ}{{\bf Z}}
\newcommand{\bfX}{{\bf X}}
\newcommand{\bfx}{{\bf x}}
\newcommand{\bfs}{{\bf s}}
\newcommand{\bfe}{{\bf e}}
\newcommand{\bfhats}{{\bf \hat{s}}}
\newcommand{\bfex}{{\langle \bfe , x \rangle }}
\newcommand{\gbar}{{\bar{g}}}
\newcommand{\Fhatn}{{\hat{F}_n}}
\newcommand{\beq}{\begin{equation}}
\newcommand{\eeq}{\end{equation}}
\newcommand{\Tau}{{\bf T}}
\newcommand{\Eta}{{\cal E}}
\newcommand{\sech}{{\rm sech}}
\newcommand{\ip}[2]{{\langle #1 , #2 \rangle }}
\newcommand{\Choose}[2]{{{\left ( \begin{array}{c} #1\\#2 \end{array} \right )}}}
\newcommand{\linSpan}{{\rm span}}
\newcommand{\dom}{{\rm dom}}
\newcommand{\gto}{{\rm G\^{a}teaux }}
\newcommand{\IF}{{\rm IF}}
\newcommand{\supp}{{\rm supp}}
\newcommand{\Pranki}{{P_{(i)}}}
\newcommand{\Prankj}{{P_{(j)}}}
\newcommand{\Prank}[1]{{P_{(#1)}}}
\newcommand{\Hrank}[1]{{H_{(#1)}}}
\newcommand{\Hranki}{{H_{(i)}}}
\newcommand{\Var}{{\bf Var}}
\newcommand{\Bias}{{\bf Bias}}
\newcommand{\jackProb}[1]{{\Prob_{(#1)}}}
\newcommand{\bootProb}{{\Prob^*}}
\newcommand{\rightWarrow}{{\begin{array}{c} {} \\ \rightarrow \\ W \end{array}}}

\begin{document}
\newtheorem{Theorem}{Theorem}
\newtheorem{Definition}{Definition}
\newtheorem{Exercise}{Exercise}
\newtheorem{Corollary}{Corollary}
\newtheorem{Lemma}{Lemma}
\newtheorem{Proof}{Proof}
\maketitle




\section{Part 10: Density Estimation. ROUGH DRAFT}
References:

\noindent
Daubechies, I. 1992. {\em Ten lectures on wavelets\/}, SIAM, Philadelphia, PA.
\\[.1in]
Donoho, D.L., I.M. Johnstone, G. Kerkyacharian, and D. Picard, 1993.
Density estimation by wavelet thresholding.
{\tt http://www-stat.stanford.edu/~donoho/Reports/1993/dens.pdf\/}
\\[.1in]
Evans, S.N. and P.B. Stark, 2002. Inverse problems as statistics,
{\em Inverse Problems\/}, {\em 18\/}, R55--R97.
\\[.1in]
Hengartner, N.W. and P.B. Stark, 1995. Finite-sample confidence envelopes for
shape-restricted densities.
{\em Ann. Stat.\/}, {\em 23\/}, pp.~525--550.
\\[.1in]
Silverman, B.W., 1990. {\em Density Estimation for Statistics and Data Analysis\/},
Chapman and Hall, London.
\\[.1in]

\subsection{Background}
Suppose we observe $\{X_j\}_{j=1}^n$ i.i.d. $F$, where $F$ has density $f$ with respect
to Lebesgue measure on the real line.
What can we learn about $f$ from these data?

Estimating $f$ can play a role in exploratory data analysis (EDA)
as a graphical summary of the data set.
In some contexts, more rigorous
estimates and inferences about $f$ and properties of $f$ such as
its value at a point $f(x_0)$, its derivative
at a point $f'(x_0)$, a Sobolev norm of $f$ such as
$\|f\|_S^2 = \int (f^2 + f'^2 + f''^2)dx$,
and the number and locations of modes of $f$,
also are interesting.
We will look at some approaches to estimating $f$, to finding confidence regious
for $f$, and to testing hypotheses
about $f$.
We will not dwell on optimality considerations.

\subsubsection{The Histogram and the Naive Estimator}
This section follows {\em Silverman\/} (1990).

Let $\hat{F}_n$ denote the empirical cdf of the data $\{X_j\}_{j=1}^n$:
\beq
    \hat{F}_n(x) = \frac{1}{n} \sum_{j=1}^n 1_{X_j \le x}.
\eeq
Although $\hat{F}_n$ is often a good estimator of $F$,
$d \hat{F}_n/dx$ is usually not a good estimator of $f = dF/dx$.
The derivative of the empirical cdf is a sum of point masses
at the observations.
It usually is not an enlightening representation of the data.

Suppose we have a collection of {\em class intervals\/} or {\em bins\/}
$\{ \calI_k = (a_k, a_{k+1}] \}_{k=1}^K$ such that every $X_j$ is in some $\calI_k$.
(Choosing the intervals to be open on the  left and closed on the right is
arbitrary; the essential point is that they be disjoint and that their
union include all the data.)
Let
\beq
    w_k = \diam(\calI_k) = a_{k+1} - a_k.
\eeq
The {\em histogram\/} of the data using these bins is
\beq
    h(x) = \frac{1}{n}\sum_{k=1}^K \frac{1}{w_k} 1_{x \in \calI_k}
    \#\left \{ \{X_j\}_{j=1}^n \cap \calI_k \right \}.
\eeq
The histogram is an estimate of $f$.
Its general appearance, including the number and locations of its modes and its smoothness,
depends strongly on the locations and widths of the bins.
It is blocky and discontinuous.
If the bin widths and locations are chosen well, its performance---in
the sense of convergence to $f$ in a norm as the sample size $n$
grows---can be reasonable.

Another estimate of $f$ derives from the definition of $f$ as the
derivative of $F$:
\beq
    f(x) = \lim_{\epsilon \rightarrow 0} \frac{F(x+\epsilon) - F(x)}{\epsilon}
         = \lim_{h \rightarrow 0} \frac{1}{2h} \Pr \{ x - h < X \le x+h \}
\eeq
One could imagine estimating $f$ by picking a small value of $h$ and taking
\begin{eqnarray}
    \hat{f}_h(x) & \equiv & \frac{1}{2h} \left ( \hat{F}_n(x+h) - \hat{F}_n(x-h) \right ) \nonumber \\
               & = & \frac{1}{2nh} \sum_{j=1}^n 1_{x-h < X_j \le x+h} \nonumber \\
               & = & \frac{1}{n}\sum_{j=1}^n \frac{1}{h} K \left ( \frac{x - X_j}{h} \right ),
\end{eqnarray}
where $K(x) = \frac{1}{2}\times1_{-1 < x \le 1}$.
This is the {\em naive density estimate\/}.
It amounts to estimating $f(x)$ by a superposition (sum) of boxcar functions
centered at the observations, each with width $2h$ and area $1/n$.
This sum is also blocky and discontinuous, but it avoids one of the arbitrary
choices in constructing a histogram: the choice of locations of the bins.
As $h \rightarrow 0$, the naive estimate converges weakly to the sum of point masses
at the data; for $h >0$, the naive estimator smooths the data.
The tuning parameter $h$ is analogous to the bin width in a histogram.
Larger values of $h$ give smoother density estimates.
Whether ``smoother'' means ``better'' depends on the true density $f$;
generally, there is a tradeoff between bias and variance: increasing the smoothness
increases the bias but decreases the variance.

It follows from the fact that $\int_{-\infty}^\infty K(x) dx = 1$ that
\begin{eqnarray}
    \int_{-\infty}^\infty \hat{f}(x) dx
        &=&
        \frac{1}{n} \sum_{j=1}^n \frac{1}{h} \int_{-\infty}^\infty K
        \left ( \frac{x - X_j}{h} \right )
        \nonumber \\
        &=& \frac{1}{n} \sum_{j=1}^n 1 = 1.
\end{eqnarray}
It follows from the fact that $K(x) \ge 0$ that $\hat{f} \ge 0$ for all $x$.
Thus $\hat{f}$ is a probability density function.

\subsection{Kernel estimates}

The two properties of the boxcar just mentioned---integrating to one and nonnegativity---hold 
whenever $K(x)$ is itself a probability density function, not just when $K$ is a 
unit-area boxcar function.
Using a smoother {\em kernel\/} function $K$, such as a Gaussian density,
leads to a smoother estimate $\hat{f}_K$.
Estimates that are linear combinations of such kernel functions centered at the
data are called {\em kernel density estimates\/}.
We denote the kernel density estimate with bandwidth (smoothing parameter) $h$ by
\beq
    \hat{f}_h(x) = \frac{1}{nh} \sum_{j=1}^n K \left ( \frac{x - X_j}{h} \right ).
\eeq
The dependence of the estimate on the kernel is not evident in the notation---the kernel
is understood from context.
Kernels are always chosen to integrate to one, but there can be asymptotic advantages
to kernels that are negative in places.
The density estimates derived using such kernels can fail to be probability densities,
because they can be negative for some values of $x$.
Typically, $K$ is chosen to be a symmetric probability density function.

There is a large body of literature on choosing $K$ and $h$ well, where ``well''
means that the estimate converges asymptotically as rapidly as possible in
some suitable norm on probability density functions.
The most common measure of performance is the mean integrated squared error (MISE):
\begin{eqnarray}
    \mbox{\rm MISE}(\hat{f}) & \equiv &  \EE \int (\hat{f}(x) - f(x))^2 dx \nonumber \\
    &=& \int \EE (\hat{f}(x) - f(x))^2 dx \nonumber \\
    &=& \int (\EE \hat{f}(x) - f(x))^2 dx + \int \Var(\hat{f}) dx.
\end{eqnarray}
The MISE is sum of the integral of the squared pointwise bias of the estimate and the
pointwise variance of the estimate.
For kernel estimates,
\beq
    \EE \hat{f}(x) = \int \frac{1}{h} K \left ( \frac{x - y}{h} \right ) f(y) dy,
\eeq
and
\beq
    \Var \hat{f}(x) = \int \frac{1}{h^2} K \left ( \frac{x-y}{h} \right )^2 f(y) dy -
    \left [ \frac{1}{h} \int K \left ( \frac{x-y}{h} \right ) f(y) dy \right ]^2.
\eeq
The expected value of $\hat{f}$ is a smoothed version of $f$, the result of
convolving $f$ with the scaled kernel.
If $f$ is itself very smooth, smoothing it by convolution with the scaled kernel does not change
its value much, and the bias of the kernel estimate is small.
But in places where $f$ varies rapidly compared with the width of the scaled kernel,
the local bias of the kernel estimate will be large.
Note that the bias depends on the kernel function and the scale (bandwidth) $h$, not
on the sample size.

The two previous expressions for bias and variance rarely lead to tractable 
computations, but good approximations
are available subject to some assumptions about $K$ and $f$.
Suppose $K$ integrates to 1, is symmetric about zero so that $\int xK(x)dx = 0$,
and has nonzero finite second central moment $\int x^2 K(x) dx = k_2 \ne 0$,
and that $f$ has as many continuous derivatives as needed.
Then, to second order in $h$,
\beq
    \Bias_h(x) \approx \frac{1}{2} h^2 f''(x) k_2.
\eeq
(See {\em Silverman\/} (1990), pp.~38ff.)
Thus
\beq
    \int \Bias_h^2 (x) dx \approx \frac{1}{4} h^4 k_2^2 \int (f'')^2(x) dx,
\eeq
confirming more quantitatively that (asymptotically)
the integrated bias depends on the smoothness
of $f$.
A similar Taylor series approximation shows that to first order in $h^{-1}$,
\beq
    \Var \hat{f}(x) \approx n^{-1}h^{-1} f(x) \int K^2(u) du,
\eeq
so
\beq
    \int \Var \hat{f}(x) dx \approx n^{-1}h^{-1} \int K^2(u) du.
\eeq
Thus shows that to reduce the integrated bias, one wants a narrow kernel,
which must have large values to satisfy $\int K = 1$,
while to reduce the integrated variance, one wants the kernel to have small
values, which requires it to be broad to satisfy $\int K = 1$.

By calculus one can show that the approximate MISE is minimized by choosing the
bandwidth to be
\beq
    h^* = n^{-1/5} k_2^{-2/5} \left [ \int K^2(u) du \right ]^{1/5}
        \left [ \int (f'')^2(x) dx \right ]^{-1/5},
\eeq
which depends on the unknown density $f$.
Note that the (approximately) optimal bandwidth for MISE decreases with $n$ as
$n^{-1/5}$.
For the (approximately) optimal bandwidth $h^*$,
\beq
    \mbox{\rm MISE} \approx n^{-4/5} \times 1.25 C(K) \left [ \int (f'')^2(x) dx \right ]^{1/5},
\eeq
where
\beq
    C(K) = k_2^{2/5} \left [ \int K^2(u) du \right ]^{4/5}
\eeq
depends only on the kernel.
The kernel that is approximately optimal for MISE thus has the smallest
possible value of $C(K)$ subject to the restrictions on the moments of $K$.
If we restrict attention to kernels that are probability density functions,
the optimal kernel is the {\em Epanechnikov kernel\/} $K_e(u)$
\beq
    K_e(u) = \frac{3}{4 \sqrt{5}} (  1 - u^2/5 )_+.
\eeq
This is the positive part of a parabola.

One can define the relative efficiency of other kernels compared with the Epanechnikov
kernel as the ratio of their values of $C(K)^{5/4}$.
Other common kernels include Tukey's Biweight (suitably normalized, this is $\frac{15}{16} (1-u^2)_+^2$),
a triangular kernel, the rectangular kernel of the naive estimate, and the Gaussian density.
Table 3.1 on p.~43 of {\em Silverman\/} (1990) shows that there is not much
variation in the efficiency: the rectangular kernel is worst, with an efficiency of about
93\%; the efficiency of the Gaussian is about 95\%; the efficiency of the triangular
kernel is about 99\%; and the efficiency of the Biweight is over 99\%.
Thus the choice of kernel can reflect other concerns, such as desired properties of $\hat{f}$
(continuity, computational complexity, and so on).

Choosing $h = h(n)$ is much more of a concern for the asymptotic behavior of the density
estimate.
To a large extent, choosing $h$ is a black art, but there are some automatic strategies
that behave well subject to some assumptions.
One of the most popular is least-squares cross-validation, which is a resampling method
related to the jackknife.
Here is a sketch of the method, following {\em Silverman\/} (1990), pp.~48ff.

The integrated squared error of a density estimate $\hat{f}$ is
\beq
    \int (\hat{f} - f)^2 dx = \int \hat{f}^2 dx - 2 \int \hat{f}f dx + \int f^2 dx.
\eeq
The last term does not involve the density estimate, so it is not in our control.
Thus it is enough to try to minimize
\beq
    R(\hat{f}) \equiv \int \hat{f}^2 dx - 2 \int \hat{f}f dx.
\eeq
Cross validation estimates $R(\hat{f}_h)$ from the data, and chooses $h$ to minimize
the estimate.
The first term in $R(\hat{f})$ can be calculated explicitly from $\hat{f}$.
Estimating the second term is the crux of the method.
By analogy to the jackknife, define the {\em leave one out\/} kernel density estimate
\beq
    \hat{f}_{h,(i)} (x) = \frac{1}{(n-1)h} \sum_{j \ne i} K \left ( \frac{x - X_j}{h} \right ) .
\eeq
Let
\beq
    M_0(h) \equiv \int \hat{f}_h^2 - \frac{2}{n} \sum_i \hat{f}_{h,(i)}(X_i).
\eeq
Let's compute the expected value of $M_0(h)$.
First note that
\begin{eqnarray}
    \EE \frac{1}{n} \sum_i \hat{f}_{h,(i)} (X_i) & = & \EE \hat{f}_{h,(1)}(X_1) \nonumber \\
    & = & \EE \int \hat{f}_{h,(1)}(x) f(x) dx \nonumber \\
    & = & \EE \int \hat{f}_{h}(x) f(x) dx.
\end{eqnarray}
The last step uses the fact that the expected value of the kernel density estimate
depends on $K$ and $h$ but not on the sample size.
Thus
\beq
    \EE R(\hat{f}_h) = \EE M_0(h),
\eeq
and $M_0(h)$ is an unbiased estimator of the ISE of $\hat{f}_h$, less the term $\int f^2$,
which does not depend on $\hat{f}_h$.
Provided $M_0(h)$ is close to $\EE M_0(h)$, choosing $h$ to minimize $M_0(h)$ should
select a good value of $h$ for minimizing the MISE of the estimate.
The form of $M_0(h)$ is not computationally efficient; simplifications are possible,
especially if $K(\cdot)$ is symmetric.
Moreover, if we use $n$ in place of $n-1$ in the denominators, we get a
similar score function $M_1(h)$ that is easier to compute:
\beq
    M_1(h) = \frac{1}{n^2h} \sum_i \sum_j K^*\left ( \frac{X_i - X_j}{h} \right ) +
        \frac{2}{nh} K(0),
\eeq
where $K^*(u) = (K \star K)(u) - 2 K(u)$ (here $\star$ denotes convolution).
The score function $M_1(h)$ can be computed very efficiently by Fourier methods; see
\S~3.5 of {\em Silverman\/} (1990).

A theorem due to Charles Stone (1984) justifies asymptotically choosing $h$ by
cross validation using the score function $M_1$.
Stone's theorem says that, subject to minor restrictions on $K$,
the ratio of the integrated squared error choosing $h$ by minimizing $M_1$ to
the integrated squared error for the best choice of $h$ given the sample $\{X_j\}$
converges to 1 with probability 1 as $n \rightarrow \infty$.

Cross validation tends to fail when the data have been discretized (binned),
because the behavior of $M_1(h)$ at small $h$ is sensitive to rounding and
discretization.
It can be rescued sometimes by restricting the optimization to a range of
values of $h$ that excludes very small values.

The MISE (or an estimate of it) is but one of many possible score functions
that could be used in a cross validation scheme.
For example, one could use the log likelihood instead, which leads to
maximizing the score function
\beq
    \mbox{CV}(h) = \frac{1}{n} \sum_{i=1}^n \log \hat{f}_{h,(i)} (X_i).
\eeq
It turns out that under strong restrictions on $f$ and $K$,
$-\mbox{CV}(h)$ is (within a constant) an unbiased estimator of the
Kullback-Leibler distance between
$\hat{f}_h$ and $f$:
\beq
    I(f, \hat{f}_h) \equiv \int f(x) \log \frac{f(x)}{\hat{f}_h(x)} dx.
\eeq
The score function $\mbox{CV}(h)$ is not resistant.

\subsection{Kernel estimates of multivariate densities}

This material is drawn from Chapter 4 of {\em Silverman\/} (1990).

Let $\{X_j\}_{j=1}^n$ each take values in $\bfR^d$, $d \ge 1$.
Let $K: \bfR^d \rightarrow \bfR$ satisfy
\beq
    \int_{\bfR^d} K(x) dx = 1.
\eeq
Typically, the kernel $K$ is a radially symmetric probability distribution
such as the standard multivariate normal, or the multivariate Epanechnikov
kernel
\beq
    K_e(x) \equiv \frac{d+2}{2c_d}(1 - \|x\|^2)_+,
\eeq
where $c_d$ is the volume of the unit sphere in $\bfR^d$:
\beq
    c_d = \int_{\bfR^d} 1_{\|x\| < 1} dx.
\eeq
The kernels
\beq
    K_2 (x) \equiv \frac{3}{\pi}(1 - \|x\|^2)_+^2
\eeq
and
\beq
    K_3(x) \equiv \frac{4}{\pi}(1 - \|x\|^2)_+^3
\eeq
have more derivatives than the Epanechnikov kernels, and thus produce
smoother density estimates; also, they are easier to compute than
the multivariate normal density.

Given a multivariate kernel function, the multivariate kernel density estimate is
\beq
    \hat{f}_h(x) \equiv \frac{1}{nh^d} \sum_{j=1}^n K \left ( \frac{x - X_j}{h} \right ),
\eeq
which is directly analogous to the univariate kernel density estimate.
The kernel density estimate is a sum of ``bumps'' centered at the observations,
each with mass $1/n$ and a common width that depends on a tuning parameter, the
bandwidth $h$.
The bandwidth $h$ is ``isotropic'' in that all coordinates are scaled in the
same way.
If the coordinates are incommensurable ({\em e.g.\/}, if the variances of different
coordinates are radically different), it can help to transform the coordinate system
before using the estimator, for example, by transforming so that the covariance
matrix of the observations is the identity matrix.
The estimate can then be transformed by the inverse change of variables to get
the density estimate in the original coordinate system.
This corresponds to the estimate
\beq
    \hat{f}_{h,S}(x) = \frac{1}{\sqrt{|S|} n h^d} \sum_{j=1}^n
        k \left ( \frac{\|x - X_j \|_{S^{-1}}^2}{h^2} \right ),
\eeq
where $\|x\|_{S^{-1}}^2 \equiv x^T S^{-1} x$, and $k(\|x\|^2) = K(x)$.

Most of the treatment of univariate kernel density estimates carries over,
{\em mutatis mutandis\/}, to the multivariate case.
For example, there is an optimal window width for minimizing the (approximate) MISE;
it depends on the smoothness of the underlying density (through $\int (\nabla^2 f)^2$)
and on the norm of the kernel and on the second moment of the kernel.
Stone's theorem shows that choosing the bandwidth by cross validation using the
score function
\beq
    M_1(h) = \frac{1}{n^2 h^d} \sum_i \sum_j K^* \left ( \frac{X_i - X_j}{h} \right )
    + \frac{2}{n h^d}K(0)
\eeq
is asymptotically optimal for MISE.

\subsubsection{The {\em Curse of Dimensionality\/}}
The difficulty of density estimation grows very rapidly as the dimension of the
sample space, $d$, increases.
For example, to get relative mean squared error at 0 to be less than 0.1
in estimating a multivariate normal density at zero using the optimal
kernel requires $n = 4$ for $d=1$, $n=19$ for $d=2$,
$n=768$ for $d=5$, and $n = 842,000$ for $d=10$
(see Table~4.2 of {\em Silverman\/}, 1990).
Partly, this is because of the behavior of the volume element in high dimensional
spaces.

Consider the unit sphere in dimension $d$.
As $d$ grows, the volume of the sphere is increasingly concentrated in
a thin shell near radius $r=1$.
As a result, regions of low density can contribute substantially to the
probability in higher dimensions, and regions of high density can
remain unsampled even for relatively large sample sizes when $d$ is large.
This makes details of the estimate matter increasingly as $d$ grows, and
makes it harder to estimate the density even where it is large as $d$ grows.

\subsection{Nearest neighbor estimates}
This section follows {\em Silverman\/} (1990), \S~5.2.
We start with the $d$-dimensional case.
For any point $t \in \bfR^d$, define $r_k(t)$ to be the
Euclidean distance from $t$ to the $k$th closest datum in
the set $\{X_j\}_{j=1}^n$.
Let $V_k(t)$ be the volume in $\bfR^d$ of a sphere of radius $r_k(t)$:
\beq
    V_k(t) = c_d r_k^d(t),
\eeq
where as before $c_d$ is the volume of the unit ball in $\bfR^d$.
The {\em nearest neighbor density estimate\/} is
\beq
    \hat{f}_k(t) \equiv \frac{k}{n V_k(t)} = \frac{k}{n c_d r_k^d(t)}.
\eeq
Usually $k$ is chosen to be small compared with $n$; $k \approx \sqrt{n}$
is typical in dimension $d=1$.
Larger values of $k$ produce smoother estimates, but the smoothness
varies locally: the effective ``window'' is narrower where the local density
of data is higher.

Why does the recipe for the nearest neighbor estimate make sense?
If the density at $t$ is $f(t)$, then in a sample of size $n$, we would
expect there to be about $n f(t) V_k(t)$ observations
in a small sphere of volume $V_k(t)$ centered at $t$.
If we set the expected number equal to the observed number and solve for
$f$, we get the nearest neighbor estimate:
\beq
    \left \{ n \hat{f}(t) V_k(t) = n \hat{f}(t) c_d r_k^d(t) = k \right \} \Rightarrow
    \left \{ \hat{f}(t) = \frac{k}{n c_d r_k^d(t)} \right \}
\eeq
At the point $t$, each datum within a distance $r_k(t)$ of $t$ contributes
$1/(n c_d r_k^d(t))$ to the density estimate---as if the density estimate
at $t$ were a kernel estimate with the kernel equal to the indicator function
of the unit ball in $\bfR^d$ divided by the volume of the ball
(so the kernel integrates to 1), with bandwidth $r_k(t)$.
Of course, this bandwidth depends on $t$ through $r_k(t)$,
so the nearest neighbor estimate can be thought of as a kernel density
estimate with spatially varying kernel width.
(The kernel width depends on the point $t$ at which the estimate is
sought, not just on the data $\{X_j\}$.
This leads to some difficulties---see below.)

Nearest neighbor estimates are not smooth: although $r_k(t)$ is continuous in
$t$, its derivative fails to exist at points where two or more data
are at distance $r_k(t)$ from $t$.
Moreover, the nearest neighbor estimate is not itself a density.
Consider what happens as $\|t\|$ grows.
When $\|t\|$ is larger than $\max_j \| X_j \|$,
$r_k(t)$ grows linearly with $\|t\|$,
so the density estimate falls off like $\|t\|^{-d}$---which has infinite
integral.
This rate of decay does not depend on how the tails of the sample
fall off.

Nearest neighbor estimates can be generalized to kernels more complicated
than indicator functions.
The generalized nearest neighbor estimate using kernel $K$ is
\beq
    \hat{f}(t) = \frac{1}{n r_k^d(t)} \sum_{j=1}^n K\left ( \frac{t - X_j}{r_k(t)} \right ).
\eeq
This reduces to the simple nearest neighborhood estimate when $K$ is the indicator
of the unit ball, scaled to have integral 1.
The tail behavior of the generalized nearest neighbor estimate depends on
details of the kernel $K$.

Even though at any fixed point, the nearest neighbor estimate is equivalent to
a kernel estimate, it is a different kernel estimate at each point.
The kernel estimate is a density because it is a linear combination of
densities, with coefficients that sum to one.
Just one ``bump'' is centered at each datum.
In contrast, with the nearest neighbor estimate, a different bump is centered at
each datum in finding the estimate for different values of $t$: the bandwidth
associated with the
contribution of the $j$th datum is a function of $t$, not just of $X_j$.

\subsubsection{Variable Kernel Method}
In contrast, the variable kernel method allows the bandwidth associated with
each {\em datum\/} to be different, but holds those bandwidths fixed as
$t$ varies.
Let $d_{jk}$ be the distance from the $X_j$ to its $k$th nearest neighbor; {\em i.e.\/},
$d_{jk} = r_{k+1}(X_j)$.
Then the variable kernel estimate is
\beq
    \hat{f} = \frac{1}{n} \sum_{j=1}^n
        \frac{K \left ( \frac{t - X_j}{hd_{jk}} \right )}{h^d d_{jk}^d}.
\eeq
As $h$ or $k$ grows, the estimate gets smoother.
This estimate centers one ``bump'' of mass $1/n$ at each datum, but the widths of the
bumps depend on the local density of observations through $d_{jk}^{-1}$.
Because of this, the estimate is itself a density if the basic kernel $K$ is a density.
When the distance to the $k$th nearest neighbor is large, the width of the bump is large.
Using $d_{jk}$ is an attempt to adapt the bandwidth to the height of the underlying
density.
However, there are better estimates of the local density to use to adjust the
bandwidth.
Of course, one can allow the bandwidth to vary in ways other than through
$d_{jk}$.

\subsubsection{Adaptive Kernel estimates}
This material is drawn from {\em Silverman\/} (1990, \S~5.3).
Instead of using $d_{jk}^{-1}$ as (proportional to) an estimate of the local
density for picking the bandwidth for the kernel centered at $X_j$, one
could use a different density estimate.
This approach leads to adaptive kernel estimates.

The idea is to make a pilot density estimate, usually highly smoothed,
and to base the bandwidth choice for the final estimate on the pilot.
Let $\tilde{f}(t)$ be a pilot density estimate for which $\min_j \tilde{f}(X_j) > 0$.
Let $\alpha \in [0, 1]$.
Let $g$ be the geometric mean of the values $\{\tilde{f}(X_j)\}_{j=1}^n$:
\beq
    g = \left ( \prod_{j=1}^n \tilde{f}(X_j) \right )^{1/n} = \exp \left \{ \frac{1}{n}
        \sum_{j=1}^n \log \tilde{f}(X_j) \right \}.
\eeq
Define
\beq
    \lambda_j = \left ( \frac{\tilde{f}(X_j)}{g} \right )^{-\alpha}.
\eeq
The adaptive kernel estimate is
\beq
    \hat{f}(t) = \frac{1}{n} \sum_{j=1}^n \frac{K
    \left ( \frac{t-X_j}{h \lambda_j} \right )}{h^d \lambda_j^d}.
\eeq
Note that $\lambda_j$ plays the role of $d_{jk}$ of the variable kernel method.
The estimate depends on a number of tuning constants: $h$, $\alpha$, and the
tuning constants of the method used to derive the pilot estimate.
The overall bandwidth $h$ plays the same role as before.
The {\em sensitivity parameter\/} $\alpha$ controls how much the bandwidth varies
as the pilot estimate varies---the rapidity of variation with $\tilde{f}$.
For $\alpha=0$, the method becomes the ordinary kernel estimate.
{\em Silverman\/} (1990) says that ``there are good reasons for setting $\alpha=1/2$.''
(See his \S~5.3.3 and reference to Abramson, 1982.)

{\em Silverman\/} (1990)
says that the fine details and smoothness of the pilot estimate don't matter much
for the final estimate, and recommends using an Epanechnikov kernel estimate
with bandwidth chosen to perform well for a standard distribution, calibrated to
have the same variance as the sample.
He does not advocate using cross validation or other computationally intensive
schemes for the pilot estimate.

{\em Silverman\/} reports simulation studies by Breiman, Meisel and Purcell (1977),
showing that with the bandwidth chosen optimally, the adaptive
kernel method performs remarkably better than the fixed kernel method,
even for tame densities such as the normal.

The overall smoothing parameter $h$ for the adaptive kernel estimate
can be chosen by least squares cross validation.

\subsubsection{Maximum Penalized Likelihood}
This section follows \S~5.4 of {\em Silverman\/} (1990);
it is connected to an approach to solving
inverse problems, which we will discuss later in the course.

Recall that the MLE of the distribution $F$ is just a sum of point masses
at the observations, each with mass $1/n$.
This is not very satisfactory as a density estimate because it is so rough.
The idea of maximum penalized likelihood is to give up some likelihood in
favor of smoothness.
We need a functional $R$ that assigns a finite positive number to some subset
of all density functions.
For example, we might take
\beq
    R(g) = \int (g'')^2 dt.
\eeq
Let $\calF$ be the set of probability density functions for which $R$ is defined and finite.
For a fixed positive number $\lambda$ (the smoothing parameter),
the penalized log-likelihood function of the density $g$ is
\beq
    \ell_\lambda (g) =\sum_{j=1}^n \log g(X_j) - \lambda R(g).
\eeq
The maximum penalized likelihood density estimate $\hat{f}$ is any density in
$\calF$ for which
\beq
    \ell_\lambda(\hat{f}) \ge \ell_\lambda(g) \;\; \forall g \in \calF.
\eeq
The maximum penalized log likelihood estimate is an ``optimal'' compromise
between maximizing the likelihood and being as smooth as possible
(in the sense of minimizing
$R$).
Estimators that optimize a tradeoff between data fit and simplicity
are quite common in many settings; they are called
{\em regularized\/} estimates.
The functional $R$ is called the {\em regularization functional\/} or 
{\em penalty functional\/}.
Measures of fidelity to the data other than the likelihood are also common.

Finding the maximum penalized likelihood estimator
can be made more tractable numerically in a variety of ways,
depending on the choice of $R$.
For example, imposing the constraint $g > 0$ is easier if one works with the square-root
of the density or with the logarithm of the density, although imposing the
other part of the constraint $g \in \calF$---that the density integrates to one---is
harder then.
Discrete approximations
to the density (such as truncated expansions in orthogonal sets of functions)
also can simplify the numerics of finding an approximate maximum
penalized likelihood estimate.
See {\em Silverman\/} (1990) for references and more detail.

The penalized maximum likelihood approach, using roughness penalties
like those described, treats the underlying density as having homogeneous
smoothness.
We will talk more about maximum penalized likelihood in the context of
nonparametric regression (function estimation).


\subsection{Confidence sets for densities with shape restrictions; lower confidence interval
for the number of modes}
Reference: {\em Hengartner and Stark\/} (1995).

\subsection{Wavelet shrinkage}
Reference: {\em Donoho et al.\/} (1993).

\subsubsection{Time-frequency localization: windowed Fourier transform and wavelets}

\subsubsection{Haar wavelets}

\subsubsection{Unconditional bases}

\subsubsection{Hard and soft thresholding}

\subsection{Inverse Problems}
Reference: {\em Evans and Stark\/} (2002).

\subsubsection{Nonparametric regression}

\subsubsection{Example: Abel's problem}
Given a frictionless bowling ball with mass $m$, a stopwatch and the ability to roll the
ball with any desired initial velocity $v_j = v_j(0)$, find
the shape of a (2-dimensional)
hill by rolling the ball with different velocities and measuring how long it
takes the ball to return.
The measurements have errors.
You can think of this as a way to survey San Francisco on a foggy day.

The {\em forward problem\/} is to predict how long it takes the ball
to return, if we know the shape of the hill.
The {\em inverse problem\/} is to use a finite set of measurements to
learn something about the shape of the hill.

Let's solve the forward problem.
It is convenient to express the shape of the hill as the
height $h(s)$ of the hill at an arc distance $s$ along the
surface of the hill from where the ball is launched.
The initial kinetic energy of the ball is
\beq
    E_j = m v_j^2(0)/2.
\eeq
As the ball ascends, its energy is conserved (the ball is frictionless),
but it is partitioned into a kinetic component and a potential
component.
The potential energy component at arc distance $s$ is
\beq
    E_{Pj}(s) = gmh(s),
\eeq
so, by conservation of energy, the kinetic energy at arc
distance $s$ is
\beq
    E_{Kj}(s) = m v_j^2(0)/2 - gmh(s).
\eeq
We can find the velocity of the ball at arc distance $s$ as follows
\begin{eqnarray}
    mv_j^2(s)/2 & = & m v_j^2(0)/2 - gmh(s) \nonumber \\
    v_j^2(s) & = & v_j^2(0) - 2gh(s) \nonumber \\
    v_j(s) & = & \sqrt{v_j^2(0) - 2gh(s)}.
\end{eqnarray}
The velocity of the ball goes to zero (and the ball starts to come back) when
$v_j^2(0) = 2gh(s)$, provided the slope of the hill does not vanish
there (then the ball would balance and never return).
Let $s_j$ satisfy $v_j^2(0) = 2gh(s_j)$.
The time it takes the ball to return is equal to the time it takes
the ball to ascend.
The time it takes the ball to come back is thus
\beq
    \tau_j = 2\int_{s=0}^{s_j} \frac{ds}{\sqrt{v_j^2(0) - 2gh(s)}}.
\eeq
This is the solution to the forward problem.
Each transit time $\tau_j$ is a nonlinear functional of the
hill profile $h(s)$.
The inverse problem is to learn something about $h(s)$ from measurements
\beq
    d_j = \tau_j + \epsilon_j, \;\; j = 1, \ldots, n,
\eeq
where $\{\epsilon_j\}_{j=1}^n$ are stochastic errors whose joint
distribution is assumed to be known---at least up to a parameter or two.
(Rarely does anybody allow the joint distribution to be more
general than a multivariate zero mean Gaussian with independent components
whose variances are known.)

It turns out that this stylized surveying problem is related to inverse
problems in seismology, helioseismology, and stereology.

\subsubsection{General framework for inverse problems}
Observe data $X$ drawn from a distribution $\Pr_\theta$ where
$\theta$ is unknown, but it is known that $\theta \in \Theta$.
Use $X$ and the constraint $\theta \in \Theta$ to learn about
$\theta$.
For example, we might want to estimate a parameter $g(\theta)$.
Assume that $\Theta$ contains at least two points; otherwise, we
know $\theta$ and $g(\theta)$ perfectly even without data.

The parameter $g(\theta)$ is {\em identifiable\/} if
\beq
    \{ g(\theta) \ne g(\eta) \} \Leftarrow \{ \Pr_\theta \ne \Pr_\eta\},\;\;
    \forall \theta, \eta \in \Theta.
\eeq
In most inverse problems, $\theta$ is not identifiable.
Little general is known about nonlinear inverse problems,
although there are particular nonlinear inverse problems, like the
surveying problem above, that are well understood.

(The ``trick'' to solving the surveying problem is to work with
$s(h)$ instead of $h(s)$, on the assumption that $h(s)$ is strictly
monotonic.
Then the forward mapping $\tau$ is a linear functional of $s(h)$,
but there are nonlinear constraints--$s(h)$ must also be monotonic.
Although hills in San Francisco are not monotonic, in the seismic
problem, there are thermodynamic arguments that the corresponding
quantity---seismic velocity as a function of radius, divided by radius---is
monotonic in Earth's core.
See, {\em e.g.\/}
Stark, P.B., 1992.  Inference in infinite-dimensional inverse problems:
discretization and duality, {\em J. Geophys. Res.\/}, {\em 97\/},
14,055--14,082.)

\subsubsection{Linear forward and inverse problems}
When the forward problem has more structure, more can be said.
The best studied class of inverse problems are linear inverse problems.

A forward problem is {\em linear\/} if the constraint set $\Theta$ is
a subset of a separable Banach space $\calT$ and for some collection $\{\kappa_j\}_{j=1}^n$
of bounded linear functionals on $\calT$,
\beq
    X_j = \kappa_j \theta + \epsilon_j,
\eeq
where $\{\epsilon_j\}_{j=1}^n$ are random errors whose distribution does not depend on
$\theta$.
Usually, such a forward problem is written
\beq
    X = K\theta + \epsilon, \;\; \theta \in \Theta.
\eeq
The corresponding {\em linear inverse problem\/} is to use the data $X$ and the constraint
$\theta \in \Theta$ to learn about $g(\theta)$.
In a linear inverse problem, the distribution of $X$ depends on $\theta$ through
$K\theta$, so if there exist $\theta, \eta \in \Theta$ such that
\beq
    K\theta = K\eta \;\mbox{ but }\; g(\theta) \ne g(\eta)
\eeq
then $g(\theta)$ is not identifiable.

Let's simplify the setup even further---we assume that $\calT$ is a Hilbert space,
that $\Theta = \calT$, and that $g$ is a {\em linear parameter\/}; that is,
\beq
    g(a \theta + b \eta) = a g(\theta) + b g(\eta)
\eeq
for all $a, b \in \bfR$ and all $\theta, \eta \in \Theta$.
The fundamental theorem of Backus and Gilbert says that then
$g(\theta)$ is identifiable if and only if $g = \sum_{j=1}^n a_j \kappa_j$
for some constants $\{a_j\}$.
In that case, if $\EE \epsilon = 0$,
$\sum_j a_j X_j$ is unbiased for $g(\theta)$, and if $e$ has covariance
matrix $\Sigma$, the MSE of this (linear) estimator is $a\cdot \Sigma \cdot a^T$.
See {\em Evans and Stark\/} (2002) for more details and proofs.

\subsection{Methods for inverse problems}
There is a huge number of methods for ``solving'' inverse problems, although
what qualifies as a solution is debatable.
These solution methods can be analyzed using traditional statistical measures
of performance, including bias and various loss criteria.
Perhaps the most important message is that without constraints, little
can be said.  The issue is finding constraints justified by the science
of the situation that still are helpful in reducing the uncertainty.
\begin{itemize}
    \item Backus-Gilbert estimation.  Finding linear functionals close, in some sense,
    to point evaluators.
    \item MLE and variants (regularization, maximum penalized likelihood, method of sieves,
        singular value truncation and weighting).
        Trading off fidelity to the data and a measure of complexity or roughness.
        With suitable assumptions, can show consistency and good rates of convergence
        if the tradeoff is tuned appropriately.
    \item Bayes estimation.  Difficult to justify in infinite-dimensional problems,
    because prior probability distributions on infinite-dimensional spaces are
    strange---it's hard to capture constraints without injecting lots of additional
    information.
    \item Minimax estimation.  Interesting papers by Donoho and others on estimating
    linear functionals or the entire model in the Hilbert space case.
    Connection between deterministic optimal recovery and minimax statistical
    estimation in the case that the errors are Gaussian, $\Theta$ is convex,
    and the parameter is a linear functional.
    \item Shrinkage estimation.  Shrinkage can improve MSE of estimates of
    high-dimensional means.  Can help with multiple Backus-Gilbert estimates.
    \item Wavelet-vaguelette shrinkage estimation. Analogue of wavelet shrinkage
    density estimation we looked at earlier.  Can outperform any linear method
    in some problems.  Papers by Donoho, Johnstone, and others.  Key idea is that
    the wavelet-vaguelette decomposition almost diagonalizes both the prior information
    and the forward problem.
    \item Strict bounds.  Analog of the method for confidence bounds on shape-restricted
    densities we looked at earlier in the class. Can get conservative joint confidence
    sets for arbitrarily many parameters of the model by finding upper and lower bounds
    on functionals over a set of models that satisfies the constraints and is in an
    infinite-dimensional confidence set based on fit to the data. Not generally
    optimal for standard measures of misfit to the data.
\end{itemize}

\end{document}


